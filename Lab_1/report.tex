\documentclass[12pt]{article}

\usepackage{geometry}
\geometry{margin=1in}
\usepackage{graphicx}
\usepackage[english]{babel}
\usepackage{fancyhdr}
\pagestyle{fancy}
\fancyhf{}
\lhead{Spring 2021 --- PHYS 421 Lab 1}
\rhead{Helen (Yeu) Chen}
\setlength{\parindent}{0cm}
\usepackage{makecell}
\usepackage{amsfonts}
\usepackage{longtable}
\usepackage{amsmath}
\usepackage{amssymb}
\usepackage{amsthm}
\graphicspath{ {./images/} }


\fancyfoot[C]{\thepage}

\begin{document}
\textbf{Hydrogen-Deuterium Spectrum Experiment}
\bigskip

\textbf{1. \textit{Introduction}}
\smallskip


In this experiment, we used the wavelength differences between hydrogen and deuterium Balmer series to calculate the mass ratio of hydrogen and deuterium.

First, we use a monochromator to scan over the sodium spectrum at a constant rate. Since the wavelength difference between the two lines of the yellow sodium doublet is known (sodium D lines), it can be used to calculate a conversion factor between wavelength difference and time difference that depends on the scanning rate we choose. Then we measured the wavelength differences between hydrogen and deuterium Balmer series (namely: $\alpha$, $\beta$, $\gamma$, $\delta$, and $\epsilon$ lines).

Using the expression of effective mass and orbital energy difference
$$\mu = \frac{mM}{M+m}, \quad E_i -E_f \propto \mu (\frac{1}{n_f^2} - \frac{1}{n_i^2}) $$
we find that 
$$\frac{\Delta \lambda_{HD}}{\lambda_H} =  \frac{\lambda_H - \lambda_D}{\lambda_H}= \frac{1/ \mu_H - 1/ \mu_D}{1/\mu_H} = \frac{1- M_H / M_D}{1+M_H/m}$$
Since $M_H/m$ and $\lambda_H$ is known and we can measure $\Delta \lambda_{HD}$, we then can solve for the desired quantity $M_H/M_D$. To improve accuracy, we plot $\Delta \lambda_{HD}$ for all lines in the Balmer series, and fit the data to find the slop (i.e  $\frac{\Delta \lambda_{HD}}{\lambda_H}$) instead of just using one single spectral line.
\bigskip

\textbf{2. \textit{Result for the mass ratio}}
\smallskip

The data plot for $\Delta \lambda_{HD}$ with respect to hydrogen wavelength is shown below (note that the fitted line is included also). 
\smallskip

\begin{center}
\includegraphics[width=16cm]{wavelength_difference}
\end{center}
\smallskip

The analysis is mainly done using python. The raw data was first smoothed using a \textit{scipy.ndimage} function called  \textit{gaussian$\_$filter1d()}. Then, by manually locating the peak and selecting proper peak width, the centroid of each peak is calculated, which then is used to calculate the peak separation (and hence the wavelength separation) shown in the plot above. The uncertainty is taken to be the absolute difference between the observed peak maximum and the calculated centroid. Finally, the data points are fitted using the python library \textit{lmfit.models}.
\smallskip

The fitted line gives a slop of $0.000257 \pm 0.000009$. Using the equation given above, the mass ratio is: $$\frac{M_H}{M_D} = 0.4957 \pm 0.0153$$
The accepted value of hydrogen-deuterium mass ration is $M_H/M_D = 0.500248$, which is only about $0.0045$ off from our measured value, so lies within the uncertainty ($0.0153$). This concluded that our measured value agree very well with the accepted value.
\bigskip

\textbf{3. \textit{Possible error}}
\smallskip

Assuming we made a mistake in measurements by pausing the wavelength scanning for 2 seconds between two sodium D lines, we want to know how would this affects our final result of the mass ratio.
\smallskip

First, we consider how this mistake will effect the scaling factor $K_{Na} = \frac{\Delta \lambda_{D}}{\Delta t_D}$. Since the time interval between sodium peaks is lengthened by our mistake
$$K_{Na} = \frac{\Delta \lambda_{D}}{\Delta t_D} \rightarrow \frac{\Delta \lambda_{D}}{\Delta t_D + 2s}$$
The scaling factor will become smaller, namely, it is scaled by $\frac{\Delta t_D}{\Delta t +2s} = 0.94026 \pm 0.00004$ (with $\Delta t_D = 31.476s \pm 0.025$ from our analysis).

Since $\Delta \lambda_{HD} = K_{Na} \Delta t_{HD}$, the new $\Delta \lambda_{HD}$ will also be scaled by the same amount. 

Recall that the mass ratio was calculated using the relation
$$ \frac{\Delta \lambda_{HD}}{\lambda_H} = \frac{1- M_H/M_D}{1+M_H/m_e}$$
so with this new scaled $\Delta \lambda_{HD}$, the new mass ratio was found to be $0.525 \pm 0.015$. Which gives a fractional increase, comparing to the values without this measurement error, of 
$$\frac{(M_H/M_D)_{incorrect}}{(M_H/M_D)_{correct}} = \frac{0.525 \pm 0.015}{0.4957 \pm 0.0153} = 1.05911 \pm 0.04461$$
So the mistake we made causes us to measure a mass ratio that is about $105 \%$ of the mass ratio without making the mistake.

\end{document}